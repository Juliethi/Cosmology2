\documentclass[12pt]{article}
\usepackage[left=2cm,top=2cm,right=2cm,bottom=2cm]{geometry}
\usepackage[parfill]{parskip}
\usepackage{amsmath}
\usepackage{graphicx}
\usepackage{listings}
\lstloadlanguages{python}
\usepackage{amssymb}
\usepackage{braket}
\usepackage{pdfpages}
\usepackage{hyperref}
\usepackage{cleveref}
\usepackage{float}
\usepackage{physics}
%\usepackage{minted}
%\usemintedstyle{pastie}


\begin{document}
    \begin{titlepage}
        \begin{center}
            \vspace*{5cm}
            
            \Huge
            \textbf{Milestone 3}
            
            \vspace*{0.5cm}
            \LARGE
            AST5220
        
            \vspace*{0.5cm}
        
            \textbf{Julie Thingwall}
        \end{center}
    \end{titlepage}

\section{Introduction}

In this projet we wish to follow in the footsteps of Petter Callin\cite{callin2006calculate} who numerically reproduces the power spectrum obtained by the CMB data. This will be done in several steps, where each step simulates the different physical processes that make up the power spectrum.

The previous milestones have consisted of calculating the background cosmology and the recombination history of the universe. This milestone will focus on calculating how the different perturbations that are responsible for all the structures we see today have evolved over time. 

This will be done by solving a coupled set of differential equations, where each equation corresponds to a perturbation in photon temperature, cold dark matter density, baryon density and the metric function/gravitational field. 

As with the first part of this project, all numerical solutions will be obtained by utilising the C++ code base provided by our lecturer, Hans Winther.


\section{Theoretical Background}
\subsection{The perturbed quantities}
In milestone 1 we solved the background cosmology, meaning we solved the time evolution of a homogenious and isotropic universe with no structures. All the structures we observe today can be sourced back to small fluctuations in temperature and density in the earliest times of the universe. 

When working with a universe that is no longer void of structures, we need the metric to represent that. When working with perturbations, one can choose to work in different coordinate systems, or gauges. For this project, we will work in the Newtonian gauge, corresponding to a metric on the form

\begin{align}\label{eq:perturbed metric}
    g_{00} &= -1-2\Psi \nonumber \\
    g_{0i} &= 0 \\
    g_{ij} &= a^2\delta_{ij} (1+2\Phi) \nonumber.
\end{align}

Here, the functions $\Psi$ and $\Phi$ represents the scalar perturbations in the gravitational field due to fluctuations in temperature and density. $\Psi$ corresponds to the Newtonien potential, while $\Phi$ corresponds to the perturbations in the spatial curvature. These perturbations are caused by and evolves with the perturbations of the different quantities in the universe. 

The time evolution of the fluctuations in photon temperature and the densities are derived by solving the Boltzman equation in Fourier space. The beauty of doing this in Fourier space is we can solve all equations individually for different wavenumbers $k$, where $k\propto\lambda^{-1}$ represents different perturbation sizes. 


In doing so, one ends up with the following set of equations for the photon temperature multipoles.


\begin{equation}\begin{aligned}\label{eq: photon multipoles dx}
    &\Theta_{0}^{\prime}=-\frac{c k}{\mathcal{H}} \Theta_{1}-\Phi^{\prime}\\
    &\begin{aligned}
    \Theta_{1}^{\prime} &=\frac{c k}{3 \mathcal{H}} \Theta_{0}-\frac{2 c k}{3 \mathcal{H}} \Theta_{2}+\frac{c k}{3 \mathcal{H}} \Psi+\tau^{\prime}\left[\Theta_{1}+\frac{1}{3} v_{b}\right] \\
    \Theta_{\ell}^{\prime} &=\frac{\ell c k}{(2 \ell+1) \mathcal{H}} \Theta_{\ell-1}-\frac{(\ell+1) c k}{(2 \ell+1) \mathcal{H}} \Theta_{\ell+1}+\tau^{\prime}\left[\Theta_{\ell}-\frac{1}{10} \Pi \delta_{\ell, 2}\right], \quad 2 \leq \ell<\ell_{\max } \\
    \Theta_{\ell}^{\prime} &=\frac{c k}{\mathcal{H}} \Theta_{\ell-1}-c \frac{\ell+1}{\mathcal{H} \eta(x)} \Theta_{\ell}+\tau^{\prime} \Theta_{\ell}, \quad \ell=\ell_{\max }
    \end{aligned}
    \end{aligned}\end{equation}

Photon temperature is the only quantity where we will be interested in more than the monopoles $(\ell = 0)$ and dipoles $(\ell = 1)$. The $\ell$'s come from the Legendre polynomial used when defining $\Theta_{\ell}$, meaning that larger $\ell$ corresponds to the small scale structures of the temperature field, as the Legendre polynomial has an increasing amount of oscillations for larger $\ell$'s. The monopole can be understood as the mean temperature at the position of an electron, the dipole is the velocity of the fluid due to Doppler effect and the quadrupole is the only relevant source for polarization signals. 


For the matter perturbations, the equivalence to the monopole and dipole perturbations is the fractional overdensity $\delta(x,k)$ and velocity $v(x,k)$ respectively. When solving the Boltzman equations for dark matter and baryons, one end up with the following set of equations

\begin{equation}\begin{aligned}\label{eq: matter perturbation dx}
    \delta_{\mathrm{CDM}}^{\prime} &=\frac{c k}{\mathcal{H}} v_{\mathrm{CDM}}-3 \Phi^{\prime} \\
    v_{\mathrm{CDM}}^{\prime} &=-v_{\mathrm{CDM}}-\frac{c k}{\mathcal{H}} \Psi \\
    \delta_{b}^{\prime} &=\frac{c k}{\mathcal{H}} v_{b}-3 \Phi^{\prime} \\
    v_{b}^{\prime} &=-v_{b}-\frac{c k}{\mathcal{H}} \Psi+\tau^{\prime} R\left(3 \Theta_{1}+v_{b}\right)
\end{aligned}\end{equation}

where $R=\frac{4 \Omega_{r 0}}{3 \Omega_{b 0} a}$. 

Finally, the equations describing the evolution of the perturbed metric functions $\Psi$ and $\Phi$ are given by 

\begin{equation}\begin{aligned}\label{eq phi psi}
    \Phi^{\prime} &=\Psi-\frac{c^{2} k^{2}}{3 \mathcal{H}^{2}} \Phi+\frac{H_{0}^{2}}{2 \mathcal{H}^{2}}\left[\Omega_{\mathrm{CDM} 0} a^{-1} \delta_{\mathrm{CDM}}+\Omega_{b 0} a^{-1} \delta_{b}+4 \Omega_{r 0} a^{-2} \Theta_{0}+4  a^{-2}\right] \\
    \Psi &=-\Phi-\frac{12 H_{0}^{2}}{c^{2} k^{2} a^{2}}\left[\Omega_{r 0} \Theta_{2}\right]
\end{aligned}\end{equation}

Note that to do this more accurately one should also include neutrino perturbations and the photon polarization. As this project is done by a mere master student, these quantities are promplty ignored!

To solve differential equation, we need some initial conditions. The initial conditions are derived from looking at very early times in the universes history, so early that $k\eta \ll 1$ for all $k$, which means that on all scales, the horizon is smaller than the wavelength of the perturbations. This yields the initial conditions

\begin{equation}\begin{aligned}\label{eq initial conditions}
    \Psi &=-\frac{1}{\frac{3}{2}+\frac{2 f_{v}}{5}} \\
    \Phi &=-\left(1+\frac{2 f_{\nu}}{5}\right) \Psi \\
    \delta_{\mathrm{CDM}} &=\delta_{b}=-\frac{3}{2} \Psi \\
    v_{\mathrm{CDM}} &=v_{b}=-\frac{c k}{2 \mathcal{H}} \Psi \\
    \Theta_{0} &=-\frac{1}{2} \Psi \\
    \Theta_{1} &=+\frac{c k}{6 \mathcal{H}} \Psi \\
    \Theta_{2} &= -\frac{20 c k}{45 \mathcal{H} \tau^{\prime}} \Theta_{1}\\
    \Theta_{\ell}&=-\frac{\ell}{2 \ell+1} \frac{c k}{\mathcal{H} \tau^{\prime}} \Theta_{\ell-1},
    \end{aligned}\end{equation}

where $f_v=0$ when not considering neutrinos. It is important to note that the value of $\Psi$ can be chosen freely when solving the equations. Here it is normalized for convenienve with regards to the next milestone.

\subsection{Tight coupling}
There is one regime to pay special attention to before one can solve the differential equations, namely the tight coupling regime. The tight coupling regime is the time period before recombination, where the mean free path of photons was extremely small due to Compton scattering. In this regime, the photons and baryons are tightly coupled behaving very akin to a fluid. Here, the only relevant multipoles are the monopole and the dipole, while all larger multipoles are supressed. 

Another thing to note in this regime is that the optical depth $\tau$ is of course, very large. The universe is optically thick. This can lead to numerical instabilites in the equations where $\tau'$ is multiplied by a small quantity. This is a problem in the expression for $v_b'$, where $\tau'$ is multiplied by the factor $(3\Theta_1 + v_b)$, which at early times is a small quantity. 

To remedy this we approximate $(3\Theta_1 + v_b)$. Doing this yields a new set of equations for $v_b'$ and $\Theta_1$, namely 

\begin{equation}\begin{aligned}\label{eq: tight coupling regime dx}
    q &=\frac{-\left[(1-R) \tau^{\prime}+(1+R) \tau^{\prime \prime}\right]\left(3 \Theta_{1}+v_{b}\right)-\frac{c k}{\mathcal{H}} \Psi+\left(1-\frac{\mathcal{H}^{\prime}}{\mathcal{H}}\right) \frac{c k}{\mathcal{H}}\left(-\Theta_{0}+2 \Theta_{2}\right)-\frac{c k}{\mathcal{H}} \Theta_{0}^{\prime}}{(1+R) \tau^{\prime}+\frac{\mathcal{H}^{\prime}}{\mathcal{H}}-1} \\
    v_{b}^{\prime} &=\frac{1}{1+R}\left[-v_{b}-\frac{c k}{\mathcal{H}} \Psi+R\left(q+\frac{c k}{\mathcal{H}}\left(-\Theta_{0}+2 \Theta_{2}\right)-\frac{c k}{\mathcal{H}} \Psi\right)\right] \\
    \Theta_{1}^{\prime} &=\frac{1}{3}\left(q-v_{b}^{\prime}\right).
\end{aligned}\end{equation}

The tight coupling regime approximation is valid until the beginning of recombination, or until $\left|\frac{d \tau}{d x}\right|<10 \cdot \min \left(1, \frac{c k}{\mathcal{H}}\right)$.

Now we have all we need to solve the equations!

\subsection{Horizons}
An important aspect to understand when analyzing how different perturbations at different scales evolve in time is the concept of horizons. A horizon defines the limit of which points in space are casually connected. One such horizon is the time variable $\eta(x)$. It is defined as the maximum distance a photon could have traveled since the big bang. Regions separated by greater comoving distances than $\eta$ simply cant communicate information between them, they are not casually linked. 

This is relevant when looking at the evolution of perturbation, because wether a perturbation is smaller or larger than the horizon will impact the way it evolves. In fact, while the horizon is an increasing quantity, the comoving wavelengths of the perturbation remains the same. This means that there exists a time $x_k$ for every $k$ where the perturbation was bigger than the horizon!

While there are many exciting consequences of super horizon perturbations, the focus here will be kept short and sweet and focus on what is needed to undertand the results of the perturbation evolution better. The most important factor is that the existence of super horizon perturbations divides the time evolution into three possible stages:

\begin{itemize}
    \item Super horizon: The evolution of the perturbations when $k\eta \ll 1$.
    \item Sub-horizon, radiation dominated: The evolution of the perturbation when it enters the horizon in the radiation dominated era.
    \item Sub-horizon, matter dominated: The perturbation enter the horizon in the matter dominated era.
\end{itemize}

Each of these epochs will affect the different types of perturbations differently. In the radiaton dominated era, the potentials are affected mainly by the photon perturbations $\Theta$. Since this era is also well before recombination, the baryons are tightly coupled with the photons and will also be affected by their perturbations. The dark matter, on the other hand, is as we know cold and non-interacting, meaning it is only affected by the gravitational potential, and thus only indirectly by the other perturbations. 

If the perturbation of a certain size $k$ enters the horizon in the radiation dominated era, we should expect the potentials to decrease rapidly. This can be understood by remembering that the evolution of a perturbation is determined by the amount of pressure versus the strength of gravity. If pressure is larger than gravity, the perturbations will not grow, but simply oscillate. Arguably, in the radiation dominated era, there is a lot of pressure, meaning the perturbations in radiation might not grow. With no growing perturbation and a universe that expands, the gravitational potential is expected to decrease. 

Knowing how the potential is affected in the radiation dominated era will help in understanding how the dark matter perturbations grow if they enter the horizon in this era. Simpy put, one shoud expect them to grow, but not as fast as in the matter dominated era because again, the pressure is higher in the radiation dominated era. 

As mentioned, the baryons and photons are tightly coupled in early times before recombination. This means that one should expect that the baryon perturbations that enter the horizon before recombination will follow the photon temperature perturbations in the beginning, before decoupling from the plasma and falling into the gravitational wells created by the dark matter. 




\section{Method and implementation}
\subsection{Code structure and parameters}
All main coding was done in the \texttt{Perturbations.cpp} file. The differntial equations were solved using the ODESolver found in the GSL library.All visualisation was done in \texttt{Milestone3\_plots.py}. 

All solutions where found in the interval $x\in[-12,0]$. We solved the equations for 100 $k$-values logarithmically spacedbefore recombination starts, or  in the interval $k\in[0.00005/\mathrm{Mpc}, 0.3/\mathrm{Mpc}]$. When plotting, we chose five $k$-values of different scales, namely $k=[0.3, 0.1,0.01,0.001,0.00005]/\mathrm{Mpc}$. All cosmological parameters were the same as in milestone 1 and 2. 

The code is structured around having one Vector3D object all\_solutions[yi][x][k] that holds all solutions to all of the equations. This was done to make it easier to iterate over all the different equations, because repeating operations on 7+ different vectors just seemed a bit tedious.

\subsection{Tight coupling regime}
Solving the equations in the tight coupling regime were only a matter of setting up the initial conditions and the right hand side of all the equations properly. We also had to find at what time  $|\tau'|<10$ and when $X_e=0.99$, which we used to define when recombination starts. To find the third condition, namely when $|\tau'|<ck/\mathcal{H}$, we looped over all x-values in a range of $\pm 2$ around the $x$-es found from the first two conditions. Lastly, we found the minimum of all three $x$s and defined that as the time when the tight coupling regime ends. 

\textit{You know.. in hindsight, this could've been more elegantly dealt with by just doing everything in one loop without the splines and the binary search for value function but I was stuborn and at this point the code works and I dont want to touch it again for fear of cursing it even more than it already is. :)}

\subsection{Full solution}
Again, we're basically just solving differential equations, so solving the full solution after tight coupling was again just a matter of correctly implementing the equations and initial conditions. The initial conditions for this end of the time evolution were simply the last calculated value in the tight coupling regime plus the given initial conditions in \cref{eq initial conditions} for the higher multipoles in the photon temperature. 

When all solutions where obtained and all bugs where sorted out \textit{which honestly were an impressive amount of bugs, I am not joking when I say this code is cursed. An entomologist would've had a field day in here.} the results were promptly splined and stored in txt-files for visualization.

\subsubsection{Plotting}
There's a lot of exciting physics going on here. To really understand what's going on, we marked the different domination eras from the background cosmology on the plots. We also marked the time of last scattering, when the visibility function peaks and when $\tau=1$. The last quantity we marked is when each perturbation enters the horizon for each wavenumber $k$. In short, like proper astrophysicists, we wanted to include as much information possible per figure! 

\section{Results}


\begin{figure}[h]
    \centering
    \includegraphics[width=0.49\textwidth]{theta0_plot.pdf} 
    \includegraphics[width=0.49\textwidth]{theta1_plot.pdf} 
    \includegraphics[width=0.49\textwidth]{theta2_plot.pdf}
    \caption{Plots showing how the monopole, dipole and quadrupole of the photon temperature perturbations evolve in time. The background colors in each plot represents the different eras of the universe, with yellow being the radiation dominated era, blue being the matter dominated era and red being the dark energy dominated era. The point of horizon entry for each $k$ is represented by the dashed lines. The black dashed line marks the point where the visibility function $\tilde{g}$ peaks. }
    \label{fig:theta}
\end{figure}

\begin{figure}[h]
    \centering
    \includegraphics[width=0.49\textwidth]{delta_plot.pdf} 
    \includegraphics[width=0.49\textwidth]{v_plot.pdf} 
    \caption{Plots showing the absolute value of the density perturbations and velocities for dark matter and baryons. For both these quantities, the baryon perturbations are plotted with a dashed line. As with \cref{fig:theta} the background colors represent the different eras and the vertical dashed lines represent the time each perturbation of size $k$ enters the horizon.}
    \label{fig:delta_v}
\end{figure}

\begin{figure}[h]
    \centering
    \includegraphics[width=0.49\textwidth]{Psi_plot.pdf} 
    \includegraphics[width=0.49\textwidth]{Phi_plot.pdf} 
    \caption{hello}
    \label{fig:phi_psi}
    \caption{Plots showing how the scalar metric perturbations evolve as a function of time. As with \cref{fig:theta} the background colors represent the different eras and the vertical dashed lines represent the time each perturbation of size $k$ enters the horizon.}
\end{figure}



\bibliography{bibtex}{}
\bibliographystyle{plain}
\end{document}