\documentclass[12pt]{article}
\usepackage[left=2cm,top=2cm,right=2cm,bottom=2cm]{geometry}
\usepackage[parfill]{parskip}
\usepackage{amsmath}
\usepackage{graphicx}
\usepackage{listings}
\lstloadlanguages{python}
\usepackage{amssymb}
\usepackage{braket}
\usepackage{pdfpages}
\usepackage{hyperref}
\usepackage{cleveref}
\usepackage{float}
\usepackage{physics}
%\usepackage{minted}
%\usemintedstyle{pastie}


\begin{document}
    \begin{titlepage}
        \begin{center}
            \vspace*{5cm}
            
            \Huge
            \textbf{Milestone 3}
            
            \vspace*{0.5cm}
            \LARGE
            AST5220
        
            \vspace*{0.5cm}
        
            \textbf{Julie Thingwall}
        \end{center}
    \end{titlepage}

\section{Introduction}

In this projet we wish to follow in the footsteps of Petter Callin\cite{callin2006calculate} who numerically reproduces the power spectrum obtained by the CMB data. This will be done in several steps, where each step simulates the different physical processes that make up the power spectrum.

\bibliography{bibtex}{}
\bibliographystyle{plain}
\end{document}