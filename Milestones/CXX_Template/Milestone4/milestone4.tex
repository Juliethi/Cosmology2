\documentclass[12pt]{article}
\usepackage[left=2cm,top=2cm,right=2cm,bottom=2cm]{geometry}
\usepackage[parfill]{parskip}
\usepackage{amsmath}
\usepackage{graphicx}
\usepackage{listings}
\lstloadlanguages{python}
\usepackage{amssymb}
\usepackage{braket}
\usepackage{pdfpages}
\usepackage{hyperref}
\usepackage{cleveref}
\usepackage{float}
\usepackage{physics}
%\usepackage{minted}
%\usemintedstyle{pastie}


\begin{document}
    \begin{titlepage}
        \begin{center}
            \vspace*{5cm}
            
            \Huge
            \textbf{Milestone 4}
            
            \vspace*{0.5cm}
            \LARGE
            AST5220
        
            \vspace*{0.5cm}
        
            \textbf{Julie Thingwall}
        \end{center}
    \end{titlepage}

\section{Introduction}
In this projet we wish to follow in the footsteps of Petter Callin\cite{callin2006calculate} who numerically reproduces the power spectrum obtained by the CMB data. This will be done in several steps, where each step simulates the different physical processes that make up the power spectrum.

The previous milestones have consisted of calculating the background cosmology, the recombination history of the universe and the evolution of the different perturbations that make up the power spectrum. This final milestone consists of actually tusing all these previously calculated qunatities to create the CMB power spectrum. 

As with previous milestones, all numerical solutions will be obtained by utilising the C++ code base provided by our lecturer, Hans Winther.

\section{Theoretical background}
\subsection{Spherical harmonics}
To understand what the power spectrum represents, we have to start with understanding spherical harmonics. The temperature field that makes up the CMB can be represented using spherical harmonics, which reads as 

\begin{equation}
    T(\hat{n})=\sum_{\ell m} a_{\ell m} Y_{\ell m}(\hat{n}).
\end{equation}

Here, $T(\hat{n})$ represents the temperature in direction $\hat{n}$, $a_{lm}$ are the spherical harmonic coefficients, and $Y_{\ell m}$ are the spherical harmonic functions themselves. Spherical harmonics are wave functions on the sphere, they are completely analogous to fourier transformations in flat space. 

The $\ell$s refer to scale, with smaller $\ell$s being bigger scales. For each $\ell$ we have $m=2\ell + 1$. Now, the CMB power spectrum shows us the expectation value for each $a_{\ell m}$, or

\begin{equation}
    C_{\ell} \equiv\left\langle\left|a_{\ell m}\right|^{2}\right\rangle=\left\langle a_{\ell m} a_{\ell m}^{*}\right\rangle.
\end{equation}

where we, for each $\ell$, take the average over all $m$. This is due to the universe being isotropic. 

\subsection{The source function and line of sight integration}
In Milestone 3, we calculated the photon temperature fluctuations $\Theta_\ell$ for $\ell \in [0,6]$. But we are really interested in the interval $\ell \in [0,1200]$, at least! Luckily, we dont have to do the calculations from the last milestone 1200 times. Thanks to Zaldarriaga  and Seljak, we can instead do something called line of sight-integration (los-integration)! This integration takes on the form

\begin{equation}\label{eq: line of sight integration}
    \Theta_{\ell}(k, x=0)=\int_{-\infty}^{0} \tilde{S}(k, x) j_{\ell}\left[k\left(\eta_{0}-\eta\right)\right] d x,
\end{equation}

where $\tilde{S}(k, x)$ is the source function, and $j_{\ell}\left[k\left(\eta_{0}-\eta\right)\right]$ are Bessel functions. The source function looks like

\begin{equation}\label{eq: source function}
    \tilde{S}(k, x)=\tilde{g}\left[\Theta_{0}+\Psi+\frac{1}{4} \Pi\right]+e^{-\tau}\left[\Psi^{\prime}-\Phi^{\prime}\right]-\frac{1}{c k} \frac{d}{d x}\left(\mathcal{H} \tilde{g} v_{b}\right)+\frac{3}{4 c^{2} k^{2}} \frac{d}{d x}\left[\mathcal{H} \frac{d}{d x}(\mathcal{H} \tilde{g} \Pi)\right].
\end{equation}

In essense, the source function explains the different physics that affect a photon on its journey from last scattering until we measure it as a CMB photon. The first term, which we can see is relevant at last scattering due to it being weighted by the visibility function, explains how a photon is affected at last scattering, when it climbs out of the gravitational wells created by the baryons before free-streaming towards us. This is called the Sachs-Wolfe term. The next term is the integrated Sachs-Wolfe term, which explains how a photon is affected when traveling throug changing gravitational potentials. The third term is a Doppler term, and the fourth term is the term-who-must-not-be-named, apparently!

With los-integration, we dont have to solve all the coupled differential equations for each ell, we only need to solve \cref{eq: line of sight integration} instead, which greatly reduces computational time!

\subsection{The temperature and matter power spectrums}
For this milestone we are interested in both the temperature power spectrum and the matter power spectrum. 

The temperature power spectrum takes on the form 

\begin{equation}\label{eq: temp photon spectrum unfinished}
    C_{\ell}=\frac{2}{\pi} \int k^{2} P_{\text {primordial }}(k) \Theta_{\ell}^{2}(k) d k
\end{equation}

where $P_{\text {primordial }}(k)$, the primordial power spectrum, looks like

\begin{equation}
\frac{k^{3}}{2 \pi^{2}} P_{\text {primordial }}(k)=A_{s}\left(\frac{k}{k_{\text {pivot }}}\right)^{n_{s}-1}.
\end{equation}

Here, $n_s$ is the spectral index for scalar perturbation, which takes on the value $n_s \approx 0.96$. $k_{pivot}$ is some scale where the amplitude is $A_s$ For our universe, we have $A_s \approx 2\times10^{-9}$ and $k_{pivot} \approx 0.05/\mathrm{Mpc}$.


The primordial power spectrum sets up the anisotropies from inflation?? idk 

Adding this back \cref{eq: temp photon spectrum unfinished}, we get

\begin{equation}\label{eq temp power spectrum}
C_{\ell}=4 \pi \int_{0}^{\infty} A_{s}\left(\frac{k}{k_{\mathrm{pivot}}}\right)^{n_{s}-1} \Theta_{\ell}^{2}(k) \frac{d k}{k}.
\end{equation}
This is the integral we wish to solve. 

For the matter power spectrum, we simply have 

\begin{equation}\label{eq matter power spectrum}
P(k, x)=\left|\Delta_{M}(k, x)\right|^{2} P_{\text {primordial }}(k)
\end{equation}

where 

\begin{equation}
\Delta_{M}(k, x) \equiv \frac{c^{2} k^{2} \Phi(k, x)}{\frac{3}{2} \Omega_{M 0} a^{-1} H_{0}^{2}}.
\end{equation}


\section{Method}
\subsection{Code structure and parameters}
All main coding was done in the \texttt{PowerSpectrum.cpp} file. The differntial equations were solved using the ODESolver found in the GSL library.All visualisation was done in \texttt{Milestone4\_plot.py}. 

All solutions where found in the interval $x\in[-12,0]$. Ideally, this intervall should have been bigger, but due to limitations from milestone 2, the optical depth was only solved for $x>-12$. We solved the equations for 1000 $k$-values logarithmically spaced, in the interval $k\in[0.00005/\mathrm{Mpc}, 0.3/\mathrm{Mpc}]$. We solved for a set of $\ell$'s in the range $\ell \in [2, 2000]$. 

\subsection{Bessel splines}
First of all, we needed to spline the bessel functions for each $\ell$. We did this logarithmically spaced with values ragning from $log(1e-8)$ to $log(4e4)$.

\subsection{LOS-integration to LOS-ODE}
To solve the line of sight integration, we wrote \cref{eq: line of sight integration} as an ordinary differential equation on the form
\begin{equation}
\frac{d \Theta_{\ell}(k, x)}{d x}=\tilde{S}(k, x) j_{\ell}\left[k\left(\eta_{0}-\eta\right)\right], \quad \Theta_{\ell}(k,-\infty)=0, 
\end{equation}

with initial conditions $\Theta_{\ell}(k, x=-\infty) \approx \Theta_{\ell}(k, x=-12) = 0$. This was solved using GSLs ODESolver.

While we mainly wanted to find $Theta_{\ell}$ to compute the temperature power spectrum, we also plotted the transfer function $\Theta_{\ell}(k)$ and the spectrum integrand $\Theta_{\ell}^2(k)/k$ for a diverse set of $\ell$-values.

\subsection{Temperature power spectrum}
We also solved \cref{eq temp power spectrum} using the ODESolver. To do this we rewrote the equation as

\begin{equation}
    \frac{dC_{\ell}}{d\mathrm{log}k}= 4\pi A_{s}\left(\frac{k}{k_{\mathrm{pivot}}}\right)^{n_{s}-1} \Theta_{\ell}^{2}(k) 
\end{equation}

where we have used the fact that $\frac{dC_{\ell}}{d\mathrm{log}k} = k\frac{dC_{\ell}}{dk}$, with initial conditions $C_{\ell}(k_{\mathrm{min}}) = 0$. 

The result from this integration was normalized by a factor $\frac{\ell(\ell + 1)}{2\pi}(10^6 \times T_{\mathrm{CMB}})^2.$

The first result of the temperature power spectrum was too large for large values of $\ell$. To investigate the source of this error, we plotted the contribution from each term of the source function separately. In doing so we found that the last term was several orders of magnitudes larger than it should be. This is probably due to some error in the derivatives of $\Theta_2$, which are the only variable unique to this last term. Unable to find this error within reasonable time, this term was dropped in the final results. Originally this term is the smallest of all the terms in the source function, most of the effect on the power spectrum stems from the Sachs-Wolfe term, so not including this last term leads to less errors than including it the way it looks now.

\subsection{Matter power spectrum}
Calculating the matter power spectrum was pretty straight forward. We only needed to calculate \cref{eq matter power spectrum}. We also wanted to find $k_{\mathrm{peak}} = \mathcal{H}(a_{eq})/c.$ This was simply done by utilizing the fact that we already have an expression for $\mathcal{H}(x)$ and a value for $x_{eq}=-8.57$ from milestone 1. 

\section{Results and discussion}
\subsection{Thetas}

In \cref{fig:thetas} we see aplot of the transfer function $\Theta_{\ell}(k)$ and the spectrum integrand $\Theta_{\ell}^2(k)/k$. I must honestly admit that I am not sure of a good way to interpret these in a physical way, so I will quote a man that is probably very proficient in astrophysics and say that they are wibbly wobbly timey wimey. :)

\begin{figure}[h]
    \centering
    \includegraphics[width=0.49\textwidth]{transerfunction.pdf}
    \includegraphics[width=0.49\textwidth]{thetaintegrand.pdf}  
    \caption{Plots showing the transfer function and spectrum integrand for a few selected values of $\ell$. The left figure shows the transfer function $\Theta_{\ell}(k)$ and the right figure shows the spectrum integrand $\Theta_{\ell}^2(k)/k$ }
    \label{fig:thetas}
\end{figure}


\subsection{Temperature power spectrum}

Now for the star of the show. In \cref{fig:cell full} we see how the power spectrum looks when including all terms in the source function. As mentioned earlier, there is a pretty severe bug somewhere in in the code calculating the perturbations. The result of including this term is a way too high amplitude for low $\ell$'s, the well known Sachs-Wolfe plateau is nowhere to be seen. 

\begin{figure}[h]
    \centering
    \includegraphics[width=0.99\textwidth]{cells_full.pdf} 
    \caption{Plot showing the temperature power spectrum when including all four terms in the source function. Here we see that the Sachs-Wolfe plateu is nonexsistent, as the power spectrum is increasing at large $\ell$'s. This means something is wrong.}
    \label{fig:cell full}
\end{figure}

To investigate this error, we plotted all the different contributions from the source function, as seen in the left figure in \cref{fig:cell debugging}. Here we see, as expected, that the biggest contributor is the Sachs-Wolfe term. We also observe the late integrated Sachs-Wolfe effect wich causes an upturn in the power spectrum for the very largest scales. However, the most important thing this figure shows is that the contribution from the last source function term. This is off by several orders of magnitude for the largest scales. 

The right figure in \cref{fig:cell debugging} shows a comparison of the power spectrum with and without this last term. Here we see that they are more or less equal for small scales, after the first peak, but they diverge completely on large scales. This shows that, in our case, excluding the quadrupole term is the right decision. 



\begin{figure}[h]
    \centering
    \includegraphics[width=0.49\textwidth]{cells_component.pdf}
    \includegraphics[width=0.49\textwidth]{cells_compare.pdf}  
    \caption{Plots showing how the different terms in the source function affect the power spectrum. The left plot shows the temperature perturbations would look if only one of the terms contributed to the power spectrum. The right plot shows the difference between including and excluding the last source function term. Here we see that the source of the error in the full solution is definietly this last term, as it raises the amplitude of the power spectrum for large scales.}
    \label{fig:cell debugging}
\end{figure}

So, in \cref{fig:cell no last term} we see a plot of only the temperature power spectrum without the errors from \cref{fig:cell full}. When analyzing the temperature power spectrum, we can start by dividing the scales that were outside or inside the horizon during recombination. 

The scales outside the horizon at recombination was never affected by the coupling between baryons or photons, or any casual physics for that matter. These perturbations were set up by inflation and remained more or less the same untill entering the horizon at later times. This is representet by the Sachs-Wolfe plateau in the power spectrum. We see an upturn in this plateau for they very large scales, this upturn stems from the late integrated Sachs-Wolfe effect. These scales entered the horizon after the universe became dark energy-dominated, so they are affected by the decaying gravitational potential due to the accelerated expansion of the universe.

For the samller scales that entered the horizon before recombination however, we see there's more physics at play. We mainly see three effects here: 
\begin{itemize}
\item Scales that enter the horizon before recombination oscillate.
\item The amplitude of these oscillations seems to alternate, with even amplitudes being higher and odd amplitudes lower.
\item Theres an overall trend of all amplitudes decreasing as $\ell$ increases.
\end{itemize}

The oscillations of the power spectrum stems from the fact that the temperature perturbations themselves oscillate with the baryons in the baryon-photon fluid. A peak in the power spectrum corresponds to a temperature perturbation that is either at a maximum or minimum in its oscillations. A trough in the power spectrum corresponds to the scales that are midway in their oscillations at recombinations. This midway-point is neither an overdensity, nor an underdensity and does then not deviate from the smooth background. 

The first peak in the power spectrum corresponds to the scale that enters the horizon just before recombination and has its first maximun at recombination, the first trough corresponds to the scale entering the horizon just a bit earlier than this, so its first maximum happens before recombination. The second peak then corresponds to a perturbation entering the horizon even earlier than this, this perturbation has had time to undergo one full oscillation before recombination, and is then at a minima. And so it goes for smaller and smaller perturbations further back in time.

The alternating amplitudes can be explained by looking further into the source of the peaks. As explained in the last two paragraphs, a peak in the temperature power spectrum corresponds to either a minimum or a maximum in the temperature perturbation. A maximum is an overdense region, and a minimum is an underdense region. In the overdense regions, one have the added effect of gravity from the baryons that increase the perturbation growth. This effect is not present in the underdense-region and thus the amplitude from overdense region will be larger. 

The last effect is the dampening of small scales. This dampening happens because the photon-baryon fluid isn't \textit{actually} a fluid. The photons scatter off of the electrons a finite amount of times. A photon will on average move a given distance per Hubble time. Photon perturbations smaller than this distance will be washed out.


\begin{figure}[h]
    \centering
    \includegraphics[width=0.99\textwidth]{cells_nolastterm.pdf} 
    \caption{Plot showing the temperature power spectrum when excluding the fourth term in the source function.}
    \label{fig:cell no last term}
\end{figure}

\subsection{Matter power spectrum}
Last, but not least, the matter power spectrum. In \cref{fig:matter power spectrum} we see a plot showing the matter power spectrum, with $k_{\mathrm{peak}} = 0.018/\mathrm{Mpc}$ marked. $k_{\mathrm{peak}}$ corresponds to the scale entering the horizon at matter/radiation equality, and marks the point where scales larger than this does not experience Meszaros suppression, while smaller scales does. 

Meszaros suppressions stems from the fact that matter perturbations hardly grow in the radiation dominated era. Perturbations that enter the horizon during this era are thss stunted, while larger perturbations that enter the horizon in later times during the matter dominated era are not affected by this suppression. 

This is clearly visible on the plot, where the power spectrum for large scale perturbations $k < k_{\mathrm{peak}}$ grow as $P(k) \propto k$, while $k > k_{\mathrm{peak}}$ decreases rapidly. The smaller the scale, the longer they have had to grow in the radiation dominated era, and the more suppressed they are.

Another interesting thing we see in the matter power spectrum are the oscillations in small scales. These stem from the same constant battle between gravity and pressure as we see in the photon temperature perturbations. Neat stuff.

\begin{figure}[h]
    \centering
    \includegraphics[width=0.99\textwidth]{matterpowerspectrum.pdf} 
    \caption{Plot showing the matter power spectrum as a function of scale $k$. Marked with a red line is the scale that enters the horizon at matter/radiation equality.}
    \label{fig:matter power spectrum}
\end{figure}

\bibliography{bibtex}{}
\bibliographystyle{plain}

\end{document}