\documentclass[12pt]{article}
\usepackage[left=2cm,top=2cm,right=2cm,bottom=2cm]{geometry}
\usepackage[parfill]{parskip}
\usepackage{amsmath}
\usepackage{graphicx}
\usepackage{listings}
\lstloadlanguages{python}
\usepackage{amssymb}
\usepackage{braket}
\usepackage{pdfpages}
\usepackage{hyperref}
\usepackage{cleveref}
\usepackage{float}
\usepackage{physics}
%\usepackage{minted}
%\usemintedstyle{pastie}


\begin{document}
    \begin{titlepage}
        \begin{center}
            \vspace*{5cm}
            
            \Huge
            \textbf{Milestone 4}
            
            \vspace*{0.5cm}
            \LARGE
            AST5220
        
            \vspace*{0.5cm}
        
            \textbf{Julie Thingwall}
        \end{center}
    \end{titlepage}

\section{Introduction}
In this projet we wish to follow in the footsteps of Petter Callin\cite{callin2006calculate} who numerically reproduces the power spectrum obtained by the CMB data. This will be done in several steps, where each step simulates the different physical processes that make up the power spectrum.

The previous milestones have consisted of calculating the background cosmology, the recombination history of the universe and the evolution of the different perturbations that make up the power spectrum. This final milestone consists of actually tusing all these previously calculated qunatities to create the CMB power spectrum. 

As with previous milestones, all numerical solutions will be obtained by utilising the C++ code base provided by our lecturer, Hans Winther.

\section{Theoretical background}
\subsection{Spherical harmonics}
To understand what the power spectrum represents, we have to start with understanding spherical harmonics. The temperature field that makes up the CMB can be represented using spherical harmonics, which reads as 

\begin{equation}
    T(\hat{n})=\sum_{\ell m} a_{\ell m} Y_{\ell m}(\hat{n}).
\end{equation}

Here, $T(\hat{n})$ represents the temperature in direction $\hat{n}$, $a_{lm}$ are the spherical harmonic coefficients, and $Y_{\ell m}$ are the spherical harmonic functions themselves. Spherical harmonics are wave functions on the sphere, they are completely analogous to fourier transformations in flat space. 

The $\ell$s refer to scale, with smaller $\ell$s being bigger scales. For each $\ell$ we have $m=2\ell + 1$. Now, the CMB power spectrum shows us the expectation value for each $a_{\ell m}$, or

\begin{equation}
    C_{\ell} \equiv\left\langle\left|a_{\ell m}\right|^{2}\right\rangle=\left\langle a_{\ell m} a_{\ell m}^{*}\right\rangle.
\end{equation}

where we, for each $\ell$, take the average over all $m$. This is due to the universe being isotropic. 

\subsection{The source function and line of sight integration}
In Milestone 3, we calculated the photon temperature fluctuations $\Theta_\ell$ for $\ell \in [0,6]$. But we are really interested in the interval $\ell \in [0,1200]$, at least! Luckily, we dont have to do the calculations from the last milestone 1200 times. Thanks to Zaldarriaga  and Seljak, we can instead do something called line of sight-integration (los-integration)! This integration takes on the form

\begin{equation}\label{eq: line of sight integration}
    \Theta_{\ell}(k, x=0)=\int_{-\infty}^{0} \tilde{S}(k, x) j_{\ell}\left[k\left(\eta_{0}-\eta\right)\right] d x,
\end{equation}

where $\tilde{S}(k, x)$ is the source function, and $j_{\ell}\left[k\left(\eta_{0}-\eta\right)\right]$ are Bessel functions. The source function looks like

\begin{equation}\label{eq: source function}
    \tilde{S}(k, x)=\tilde{g}\left[\Theta_{0}+\Psi+\frac{1}{4} \Pi\right]+e^{-\tau}\left[\Psi^{\prime}-\Phi^{\prime}\right]-\frac{1}{c k} \frac{d}{d x}\left(\mathcal{H} \tilde{g} v_{b}\right)+\frac{3}{4 c^{2} k^{2}} \frac{d}{d x}\left[\mathcal{H} \frac{d}{d x}(\mathcal{H} \tilde{g} \Pi)\right].
\end{equation}

In essense, the source function explains the different physics that affect a photon on its journey from last scattering until we measure it as a CMB photon. The first term, which we can see is relevant at last scattering due to it being weighted by the visibility function, explains how a photon is affected at last scattering, when it climbs out of the gravitational wells created by the baryons before free-streaming towards us. This is called the Sachs-Wolfe term. The next term is the integrated Sachs-Wolfe term, which explains how a photon is affected when traveling throug changing gravitational potentials. The third term is a Doppler term, and the fourth term is the term-who-must-not-be-named, apparently!

With los-integration, we dont have to solve all the coupled differential equations for each ell, we only need to solve \cref{eq: line of sight integration} instead, which greatly reduces computational time!

\subsection{The temperature and matter power spectrums}
For this milestone we are interested in both the temperature power spectrum and the matter power spectrum. 

The temperature power spectrum takes on the form 

\begin{equation}\label{eq: temp photon spectrum unfinished}
    C_{\ell}=\frac{2}{\pi} \int k^{2} P_{\text {primordial }}(k) \Theta_{\ell}^{2}(k) d k
\end{equation}

where $P_{\text {primordial }}(k)$, the primordial power spectrum, looks like

\begin{equation}
\frac{k^{3}}{2 \pi^{2}} P_{\text {primordial }}(k)=A_{s}\left(\frac{k}{k_{\text {pivot }}}\right)^{n_{s}-1}.
\end{equation}

Here, $n_s$ is the spectral index for scalar perturbation, which takes on the value $n_s \approx 0.96$. $k_{pivot}$ is some scale where the amplitude is $A_s$ For our universe, we have $A_s \approx 2\times10^{-9}$ and $k_{pivot} \approx 0.05/\mathrm{Mpc}$.


The primordial power spectrum sets up the anisotropies from inflation?? idk 

Adding this back \cref{eq: temp photon spectrum unfinished}, we get

\begin{equation}\label{eq temp power spectrum}
C_{\ell}=4 \pi \int_{0}^{\infty} A_{s}\left(\frac{k}{k_{\mathrm{pivot}}}\right)^{n_{s}-1} \Theta_{\ell}^{2}(k) \frac{d k}{k}.
\end{equation}
This is the integral we wish to solve. 

For the matter power spectrum, we simply have 

\begin{equation}\label{eq matter power spectrum}
P(k, x)=\left|\Delta_{M}(k, x)\right|^{2} P_{\text {primordial }}(k)
\end{equation}

where 

\begin{equation}
\Delta_{M}(k, x) \equiv \frac{c^{2} k^{2} \Phi(k, x)}{\frac{3}{2} \Omega_{M 0} a^{-1} H_{0}^{2}}.
\end{equation}


\section{Method}
\subsection{Code structure and parameters}
All main coding was done in the \texttt{PowerSpectrum.cpp} file. The differntial equations were solved using the ODESolver found in the GSL library.All visualisation was done in \texttt{Milestone4\_plot.py}. 

All solutions where found in the interval $x\in[-12,0]$. Ideally, this intervall should have been bigger, but due to limitations from milestone 2, the optical depth was only solved for $x>-12$. We solved the equations for 1000 $k$-values logarithmically spaced, in the interval $k\in[0.00005/\mathrm{Mpc}, 0.3/\mathrm{Mpc}]$. We solved for a set of $\ell$'s in the range $\ell \in [2, 2000]$. 

\subsection{Bessel splines}
First of all, we needed to spline the bessel functions for each $\ell$. We did this logarithmically spaced with values ragning from $log(1e-8)$ to $log(4e4)$.

\subsection{LOS-integration to LOS-ODE}
To solve the line of sight integration, we wrote \cref{eq: line of sight integration} as an ordinary differential equation on the form
\begin{equation}
\frac{d \Theta_{\ell}(k, x)}{d x}=\tilde{S}(k, x) j_{\ell}\left[k\left(\eta_{0}-\eta\right)\right], \quad \Theta_{\ell}(k,-\infty)=0, 
\end{equation}

with initial conditions $\Theta_{\ell}(k, x=-\infty) \approx \Theta_{\ell}(k, x=-12) = 0$. This was solved using GSLs ODESolver.

While we mainly wanted to find $Theta_{\ell}$ to compute the temperature power spectrum, we also plotted the transfer function $\Theta_{\ell}(k)$ and the spectrum integrand $\Theta_{\ell}^2(k)/k$ for a diverse set of $\ell$-values.

\subsection{Temperature power spectrum}
We also solved \cref{eq temp power spectrum} using the ODESolver. To do this we rewrote the equation as

\begin{equation}
    \frac{dC_{\ell}}{d\mathrm{log}k}= 4\pi A_{s}\left(\frac{k}{k_{\mathrm{pivot}}}\right)^{n_{s}-1} \Theta_{\ell}^{2}(k) 
\end{equation}

where we have used the fact that $\frac{dC_{\ell}}{d\mathrm{log}k} = k\frac{dC_{\ell}}{dk}$, with initial conditions $C_{\ell}(k_{\mathrm{min}}) = 0$. 

The result from this integration was normalized by a factor $\frac{\ell(\ell + 1)}{2\pi}(10^6 \times T_{\mathrm{CMB}})^2.$

The first result of the temperature power spectrum was too large for large values of $\ell$. To investigate the source of this error, we plotted the contribution from each term of the source function separately. In doing so we found that the last term was several orders of magnitudes larger than it should be. This is probably due to some error in the derivatives of $\Theta_2$, which are the only variable unique to this last term. Unable to find this error within reasonable time, this term was dropped in the final results. Originally this term is the smallest of all the terms in the source function, most of the effect on the power spectrum stems from the Sachs-Wolfe term, so not including this last term leads to less errors than including it the way it looks now.

\subsection{Matter power spectrum}
Calculating the matter power spectrum was pretty straight forward. We only needed to calculate \cref{eq matter power spectrum}. We also wanted to find $k_{\mathrm{peak}} = \mathcal{H}(a_{eq}).$ This was done by blablablaba

\section{Results}
\subsection{Thetas}

\begin{figure}[h]
    \centering
    \includegraphics[width=0.99\textwidth]{transerfunction.pdf} 
    \label{fig:transferfunction}
    \caption{Plot showing the transfer function for a few selected values of $\ell$. The $\Theta$'s shown here is calculated }
\end{figure}

\begin{figure}[h]
    \centering
    \includegraphics[width=0.99\textwidth]{thetaintegrand.pdf} 
    \label{fig:spectrum integrand}
    \caption{Plot showing the spectrum integrand for a few selected values of $\ell$.}
\end{figure}


\subsection{Temperature power spectrum}

\begin{figure}[h]
    \centering
    \includegraphics[width=0.99\textwidth]{cells_full.pdf} 
    \label{fig:cell full}
    \caption{Plot showing the temperature power spectrum when including all four terms in the source function. Here we see that the Sachs-Wolfe plateu is nonexsistent, as the power spectrum is increasing at large $\ell$'s. This means something is wrong.}
\end{figure}

\begin{figure}[h]
    \centering
    \includegraphics[width=0.49\textwidth]{cells_component.pdf}
    \includegraphics[width=0.49\textwidth]{cells_compare.pdf}  
    \label{fig:cell debugging}
    \caption{Plots showing how the different terms in the source function affect the power spectrum. The left plot shows the temperature perturbations would look if only one of the terms contributed to the power spectrum. The right plot shows the difference between including and excluding the last source function term. Here we see that the source of the error in the full solution is definietly this last term, as it raises the amplitude of the power spectrum for large scales.}
\end{figure}

\begin{figure}[h]
    \centering
    \includegraphics[width=0.99\textwidth]{cells_nolastterm.pdf} 
    \label{fig:cell debugging}
    \caption{Plot showing the temperature power spectrum when excluding the fourth term in the source function.}
\end{figure}

\subsection{Matter power spectrum}

\bibliography{bibtex}{}
\bibliographystyle{plain}

\end{document}