\documentclass[12pt]{article}
\usepackage[left=2cm,top=2cm,right=2cm,bottom=2cm]{geometry}
\usepackage[parfill]{parskip}
\usepackage{amsmath}
\usepackage{graphicx}
\usepackage{listings}
\lstloadlanguages{python}
\usepackage{amssymb}
\usepackage{braket}
\usepackage{pdfpages}
\usepackage{hyperref}
\usepackage{cleveref}
\usepackage{float}
\usepackage{physics}
%\usepackage{minted}
%\usemintedstyle{pastie}


\begin{document}
    \begin{titlepage}
        \begin{center}
            \vspace*{5cm}
            
            \Huge
            \textbf{Milestone 2}
            
            \vspace*{0.5cm}
            \LARGE
            AST5220
        
            \vspace*{0.5cm}
        
            \textbf{Julie Thingwall}
        \end{center}
    \end{titlepage}

\section{Introduction}
In this projet we wish to follow in the footsteps of Petter Callin\cite{callin2006calculate} who numerically reproduces the power spectrum obtained by the CMB data. This will be done in several steps, where each step simulates the different physical processes that make up the power spectrum.

The first milestone consisted of calculating the background cosmology of the whole universe. For this milestone, the goal is to simulate the physics behind recombination. This will be done by computing how the optical depth and visibility function evolves in time. In doing so, we will also need to solve the Saha equation and the Peebles equations.

As with the first part of this project, all numerical solutions will be obtained by utilising the C++ code base provided by our lecturer, Hans Winther.

\section{Theorethical Background}
\subsection{Optical depth}
As mentioned, the main goal of this project is to compute how the optical depth evolves in time. In short terms, the optical depth explains wether a medium is optically thin, or optically thick. If you send a beam of light through an optically thick medium, the light will be scattered in all directions, and not much will pass through, which of course means that if you do the same throuhg an optically thin medium, most, if not all of the light will pass through. 

The intensitiy $I$ of such a beam is given by $I(x)=I_0e^{-\tau(x)}$ where $I_0$ is the initial intensity at the source and $I(x)$ is the intensity measured after some distance $x$. Here, $\tau$ represents the optical depth. If $\tau \ll 1$, we say the medium is optically thin, and if $\tau \gg 1$, the medium is thick. $\tau \approx 1$ is the transition between these two states.

The optical depth can be expressed as an integral
\begin{equation}\label{eq:opticaldepth integral}
    \tau(\eta)=\int_{\eta}^{\eta_{0}} n_{e} \sigma_{T} a d \eta^{\prime}
\end{equation}

or as an ordinary differential equation

\begin{equation}\label{eq: opticaldepth ode}
    \tau^{\prime}=\frac{d \tau}{d x}=-\frac{n_{e} \sigma_{T}}{H}.
\end{equation}

Here, $n_e$ is the electron density, $\sigma_T = \frac{8 \pi}{3} \frac{\alpha^{2} \hbar^{2}}{m_{e}^{2} c^{2}}$ is the Thompson scattering. $a$ and $H$ is of course the scale factor and Hubble parameter. As with milestone 1, $\eta$ and $x$ are our two time variables of interest, namely the conformal time and the log-scaled scale factor $x = \ln{a}$ respectively.

The reason for why the optical depth is interesting for us to calculate in regards to recombination because the moment of recombination is really just the moment where the universe goes from being optically thick to optically thin. Before recombination, the universe is hot and dense, too hot for even hydrogen to form, meaning there was a lot of free electrons scattered about for the photons to bounce off of, resulting in an optically thick universe. When the universe cooled down, all the free protons and electrons could form into neutral hydrogen, and the photons could travel freely, making the universe optically thin. 


\subsection{Electron density}
To calculate the optical depth given in \cref{eq: opticaldepth ode} we need to know how the electron density $n_e = n_e(\eta)$ behaves in time. To do this, we will calculate the fractional electron density

\begin{equation}\label{eq: x_e}
    X_{e} = \frac{n_{e}}{n_{H}}
\end{equation}

where 

\begin{equation}\label{eq: proton density}
    n_{H}=n_{b} \approx \frac{\rho_{b}}{m_{H}}=\frac{\Omega_{b} \rho_{c}}{m_{H} a^{3}}
\end{equation}

is the proton density under the assumption that the only baryons in the universe are protons. 

$X_e$ can be solved for in different ways. When the universe is in thermodynamic equillibrium, that is when $X_e \approx 1$, we can use the Saha equation. When the universe begins to cool, during and after recombination, the Saha equation falls short, and we will need to solve the Peebles equation instead. For simplicity we will use the Saha equation in the regime when $X_e > 0.99$ and Peebles equation when $X_e < 0.99$

\subsubsection{Saha equation}
As mentioned, the Saha equation works perfectly well under the assumption that the universe is in thermodynamic equilibrium. Before recombination, in the dense soup of electrons and protons, we can arguably say that it is. 

The Saha equation is defined as 

\begin{equation}\label{eq: saha equation}
    \frac{X_{e}^{2}}{1-X_{e}}=\frac{1}{n_{b}}\left(\frac{m_{e} T_{b}}{2 \pi}\right)^{3 / 2} e^{-\epsilon_{0} / T_{b}},
\end{equation}
where $m_e$ is the electron mass and $T_b$ is the baryon temperature. For simplicity, we assume that the baryon temperature follows the photon temperature, that is $T_b = T_r = T_{cmb}/a$. 

\subsubsection{Peebles equation}
When we can no longer trust the solution from the Saha equation, we will use the Peebles equation. The Peebles equation takes on the form 
\begin{equation}\label{eq: peebles equation}
    \frac{d X_{e}}{d x}=\frac{C_{r}\left(T_{b}\right)}{H}\left[\beta\left(T_{b}\right)\left(1-X_{e}\right)-n_{H} \alpha^{(2)}\left(T_{b}\right) X_{e}^{2}\right],
\end{equation}
where

\begin{align*}\label{eq: peebles components}
        C_{r}\left(T_{b}\right) &=\frac{\Lambda_{2 s \rightarrow 1 s}+\Lambda_{\alpha}}{\Lambda_{2 s \rightarrow 1 s}+\Lambda_{\alpha}+\beta^{(2)}\left(T_{b}\right)} \\
        \Lambda_{2 s \rightarrow 1 s} &=8.227 \mathrm{s}^{-1} \\
        \Lambda_{\alpha} &=H \frac{\left(3 \epsilon_{0}\right)^{3}}{(8 \pi)^{2} n_{1 s}} \\
        n_{1 s} &=\left(1-X_{e}\right) n_{H} \\
        \beta^{(2)}\left(T_{b}\right) &=\beta\left(T_{b}\right) e^{3 \epsilon_{0} / 4 T_{b}} \\
        \beta\left(T_{b}\right) &=\alpha^{(2)}\left(T_{b}\right)\left(\frac{m_{e} T_{b}}{2 \pi}\right)^{3 / 2} e^{-\epsilon_{0} / T_{b}} \\
        \alpha^{(2)}\left(T_{b}\right) &=\frac{64 \pi}{\sqrt{27 \pi}} \frac{\alpha^{2}}{m_{e}^{2}} \sqrt{\frac{\epsilon_{0}}{T_{b}}} \phi_{2}\left(T_{b}\right) \\
        \phi_{2}\left(T_{b}\right) &=0.448 \ln \left(\epsilon_{0} / T_{b}\right).
\end{align*}

Spooky stuff. Write more about this later


\subsection{Visibility function}
\begin{equation}
    \tilde{g}(x)=-\tau^{\prime} e^{-\tau}, \quad \int_{-\infty}^{0} \tilde{g}(x) d x=1
\end{equation}

\section{Method and implementation}
\subsection{Code structure}
For this milestone, all main coding was done in the file \texttt{RecombinationHistort.cpp}. We again utilized the Spline- and ODEsolver-methods from the GSL package. All visualization of the data was done using Python, and can be found in \texttt{milestone2\_plots.py}.

\subsection{Reintroduction of constants}
As most phycisists, we like to keep things simple, meaning that we don't like to drag around constants unless we absolutely have to. This means that in all equations above we have used natural units, meaning that $c = \hbar = k_b = 1$. Now, had we implemented the equations as they stand now, it would lead to some problems. First of all, most of our equations contains an exponential, and the argument of an exponential has to be unitless. Second of all, we're calculatuing $X_e$, which is unitless, which means that the equations used to find this quantity will also need to be unitless.

To achieve this we reintroduced the constants $c, \hbar, k_b$ in the equations so that they have the right dimentionality again. This can be done in several clever and rigorous ways. Or it can be done by making educated guesses, throwing some constants at the equations and see what sticks. We, of course, did the latter, anything else would have been silly... \textit{it would also probably have been much quicker}.

The units for the three constants we want to reintroduce are as following:
\begin{align*}
    \left[\hbar\right] &= Js = \frac{kgm^2}{s} \\
    \left[c\right] &= \frac{m}{s} \\
    \left[k_b\right] &= \frac{J}{K} = \frac{kgm^2}{Ks^2}.
\end{align*}

The very messy process of fixing the dimentionality will be fully explained with the Saha equation, by then both the reader and the author will probably be tired of reading and writing about how "we tried this and that until it worked", so the proccess behind the Peebles equation will be kept short and sweet with a focus on the main logic behind all educated guesses.

\subsubsection{Saha equation}
As a reminder, the Saha equation looks like 
\begin{equation}\label{eq: saha equation again}
    \frac{X_{e}^{2}}{1-X_{e}}=\frac{1}{n_{b}}\left(\frac{m_{e} T_{b}}{2 \pi}\right)^{3 / 2} e^{-\epsilon_{0} / T_{b}},
\end{equation}.

where the left hand side if, of course, unitless. Now, the units for each constant on the right hand side is $[n_b] = 1/m^3$, $[m_e]=kg$, $[T_b] = K$ and $[\epsilon_0] = eV = J$. 

First and foremost, we know that every $\epsilon_0/T_b$ should be $\epsilon_0/k_b T_b$, as both $\epsilon_0$ and $k_b T_b$ has units energy. The exponential in the expression is thus dealt with. 

The remaining units to deal with are then as following
\begin{align}
    \frac{1}{n_{b}}\left(\frac{m_{e} T_{b}}{2 \pi}\right)^{3 / 2} = m^3(kgK)^{3/2}.
\end{align}

To make this easier to work with, we squared everything, resulting in
\begin{align}
m^6K^3kg^3,
\end{align}
meaning we have to find a combination of constants with the units $m^{-6}kg^{-3}K^{-3}$.

So, the first obvious choice here is that we need the Boltzmann constant $k_b$, as that is the only constant with temperature as part of its units. Specifically, we need $[k_b^3] = kg^3m^6/K^3s^6$. This is where a more rigorous approach would have been more appropriate, and would have been easier to replicate on paper, but basically the process from here on out consisted of just observing that we at least need $kg^-{6}$ and $m^{-12}$ as a start, and then just staring really hard at what we had to work with until it became obvious that we needed to divide by $\hbar^6$. This yielded 

\begin{align}
    \frac{k_b^3}{\hbar^6} &= \frac{kg^3m^6}{K^3s^6} \frac{s^6}{kg^6m^{12}} \\
    \frac{k_b^3}{\hbar^6} &= \frac{1}{kg^3m^6K^3},
\end{align} 
which is what we wanted. Taking the square root of this and inserting it back into \cref{eq: saha equation again} gives

\begin{equation}\label{eq: final saha equation}
    \frac{X_{e}^{2}}{1-X_{e}}=\frac{1}{n_{b}\hbar^3}\left(\frac{m_{e} T_{b}k_b}{2 \pi}\right)^{3 / 2} e^{-\epsilon_{0} /k_b T_{b}}.
\end{equation}


\subsubsection{Peebles equation}
Now, to say that we followed this exact procedure with the Peebles equation would be an oversimplification, given the more complicated nature of the equation with all it's different products. But in essence, that's what was done with each separate subequation. The important part was to keep track on which subequations are coupled and to work in a clever order from there. This process will be explained as briefly as possible as to not give the reader a headache! 

Again, as a reminder, the Peebles equation takes on the form 
\begin{equation}\label{eq: peebles equation again}
    \frac{d X_{e}}{d x}=\frac{C_{r}\left(T_{b}\right)}{H}\left[\beta\left(T_{b}\right)\left(1-X_{e}\right)-n_{H} \alpha^{(2)}\left(T_{b}\right) X_{e}^{2}\right],
\end{equation}
where

\begin{align*}\label{eq: peebles components again}
        C_{r}\left(T_{b}\right) &=\frac{\Lambda_{2 s \rightarrow 1 s}+\Lambda_{\alpha}}{\Lambda_{2 s \rightarrow 1 s}+\Lambda_{\alpha}+\beta^{(2)}\left(T_{b}\right)} \\
        \Lambda_{2 s \rightarrow 1 s} &=8.227 \mathrm{s}^{-1} \\
        \Lambda_{\alpha} &=H \frac{\left(3 \epsilon_{0}\right)^{3}}{(8 \pi)^{2} n_{1 s}} \\
        n_{1 s} &=\left(1-X_{e}\right) n_{H} \\
        \beta^{(2)}\left(T_{b}\right) &=\beta\left(T_{b}\right) e^{3 \epsilon_{0} / 4 T_{b}} \\
        \beta\left(T_{b}\right) &=\alpha^{(2)}\left(T_{b}\right)\left(\frac{m_{e} T_{b}}{2 \pi}\right)^{3 / 2} e^{-\epsilon_{0} / T_{b}} \\
        \alpha^{(2)}\left(T_{b}\right) &=\frac{64 \pi}{\sqrt{27 \pi}} \frac{\alpha^{2}}{m_{e}^{2}} \sqrt{\frac{\epsilon_{0}}{T_{b}}} \phi_{2}\left(T_{b}\right) \\
        \phi_{2}\left(T_{b}\right) &=0.448 \ln \left(\epsilon_{0} / T_{b}\right).
\end{align*}
are the aforementioned subequations.

The end goal is to make $\frac{C_r}{H}\beta(T_b)(1-X_e)$ and $\frac{C_r}{H}n_H\alpha^{(2)}(T_b)X_e^2$ unitless. First of all, all $\epsilon_0/T_b$ are in fact $\epsilon_0/k_b T_b$ as before. 

Now comes the fun part. We started looking at $C_r$. We know that $[\Lambda_{2s\rightarrow1s}] = s^{-1}$, meaning that the easiest thing to do is to make all other parts of this equation have units $s^{-1}$ as well. Achieving this would yield $C_r$ unitless, making everything down the line easier. 

To do this, we first looked at $\Lambda_{\alpha}$. As it stands, this has units $[\Lambda_{\alpha}]=s^{-1}J^3m^3$. Multiplying this with $(\hbar c)^{-3}$ does the trick. This results in 
\begin{equation*}
    \Lambda_{\alpha}= H\frac{(3\epsilon_0)^3}{(8\pi)^2 n_{1s}}(\hbar c)^{-3}, [s^{-1}].
\end{equation*}

Furthermore, $\beta^{(2)}$ should also have units $s^{-1}$. To achieve this, we added a $k_b$ in the exponential and decided to look at $\beta(T_b)$ immediately, yielding 

\begin{equation*}
    \beta^{(2)}(T_b) = \beta(T_b)e^{3\epsilon_0/4k_bT_b}, [s^{-1}].
\end{equation*}

This puts a constraint on $\beta(T_b)$, namely that it should also have units $s^{-1}$. Looking back at the \cref{eq: peebles equation again}, this makes sense, as $\frac{C_r}{H}\beta(T_b)(1-X_e)$ should be unitless, and $[H] = s^{-1}$. This means we're on the right track! But this is also where it's important to pay attention to the end goal. $\beta(T_b)$ contains the function $\alpha^{(2)}(T_b)$, which we need to make sure have the right units because $\frac{C_r}{H}n_H\alpha^{(2)}(T_b)X_e^2$ needs to be unitless. Thus, we dealt with $\alpha^{(2)}$ first. 

Now, after inserting all $k_b$ we need in both $\alpha^{(2)}$ and $\phi_2$, we have $[\alpha^{(2)}] = kg^{-2}$. Again, we want $\alpha^{(2)}$ to cancel out the units of $[\frac{C_r}{H}n_HX_e^2] = s/m^3$. This means had to find a combination of constants giving $[kg^2m^3/s]$, which $\hbar^2/c$ does. Inserting this into $\alpha^{(2)}$ yields

\begin{equation*}
    \alpha^{(2)}(T_b)  =\frac{64 \pi}{\sqrt{27 \pi}} \frac{(\hbar\alpha)^{2}}{cm_{e}^{2}} \sqrt{\frac{\epsilon_{0}}{k_bT_{b}}} \phi_{2}\left(T_{b}\right), [m^3/s]
\end{equation*}

with

\begin{equation*}
    \phi_2(T_b) = 0.448\ln(\epsilon_0/k_b T_b).
\end{equation*}


Almost finished! Now all that's left is to get $[\beta(T_b)] = s^{-1}$. With the units for $\alpha^{(2)}$ known, we also know that $[\beta(T_b)] = m^3/s\cdot\left(kgK\right)^{3/2}$. This looked awfully familiar, so we just inserted the solution already found from the Saha equation, namely that $k_b^{3/2}/\hbar^3$, which returns

\begin{equation*}
    \beta(T_b) = \frac{\alpha^{(2)}}{\hbar^3} \left(\frac{m_e k_b T_b}{2\pi}\right)^{3/2} e^{-\epsilon_0/k_b T_b}, [s^{-1}].
\end{equation*}

test test test

Now all equations has the proper dimentionality and can be implemented in the code. \textit{Phew. Now that was neither brief, short or sweet, and it might be a bit more headache-inducing than I meant for it to be.}

\subsection{Solving the Saha equation}
\section{Results}

\bibliography{bibtex}{}
\bibliographystyle{plain}
\end{document}